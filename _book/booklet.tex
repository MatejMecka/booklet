% Options for packages loaded elsewhere
\PassOptionsToPackage{unicode}{hyperref}
\PassOptionsToPackage{hyphens}{url}
%
\documentclass[
]{book}
\usepackage{lmodern}
\usepackage{amsmath}
\usepackage{ifxetex,ifluatex}
\ifnum 0\ifxetex 1\fi\ifluatex 1\fi=0 % if pdftex
  \usepackage[T1]{fontenc}
  \usepackage[utf8]{inputenc}
  \usepackage{textcomp} % provide euro and other symbols
  \usepackage{amssymb}
\else % if luatex or xetex
  \usepackage{unicode-math}
  \defaultfontfeatures{Scale=MatchLowercase}
  \defaultfontfeatures[\rmfamily]{Ligatures=TeX,Scale=1}
\fi
% Use upquote if available, for straight quotes in verbatim environments
\IfFileExists{upquote.sty}{\usepackage{upquote}}{}
\IfFileExists{microtype.sty}{% use microtype if available
  \usepackage[]{microtype}
  \UseMicrotypeSet[protrusion]{basicmath} % disable protrusion for tt fonts
}{}
\makeatletter
\@ifundefined{KOMAClassName}{% if non-KOMA class
  \IfFileExists{parskip.sty}{%
    \usepackage{parskip}
  }{% else
    \setlength{\parindent}{0pt}
    \setlength{\parskip}{6pt plus 2pt minus 1pt}}
}{% if KOMA class
  \KOMAoptions{parskip=half}}
\makeatother
\usepackage{xcolor}
\IfFileExists{xurl.sty}{\usepackage{xurl}}{} % add URL line breaks if available
\IfFileExists{bookmark.sty}{\usepackage{bookmark}}{\usepackage{hyperref}}
\hypersetup{
  pdftitle={Препораки за репродуцибилно анализирање на податоци},
  pdfauthor={Теофил Наков, Новица Наков},
  hidelinks,
  pdfcreator={LaTeX via pandoc}}
\urlstyle{same} % disable monospaced font for URLs
\usepackage{longtable,booktabs}
% Correct order of tables after \paragraph or \subparagraph
\usepackage{etoolbox}
\makeatletter
\patchcmd\longtable{\par}{\if@noskipsec\mbox{}\fi\par}{}{}
\makeatother
% Allow footnotes in longtable head/foot
\IfFileExists{footnotehyper.sty}{\usepackage{footnotehyper}}{\usepackage{footnote}}
\makesavenoteenv{longtable}
\usepackage{graphicx}
\makeatletter
\def\maxwidth{\ifdim\Gin@nat@width>\linewidth\linewidth\else\Gin@nat@width\fi}
\def\maxheight{\ifdim\Gin@nat@height>\textheight\textheight\else\Gin@nat@height\fi}
\makeatother
% Scale images if necessary, so that they will not overflow the page
% margins by default, and it is still possible to overwrite the defaults
% using explicit options in \includegraphics[width, height, ...]{}
\setkeys{Gin}{width=\maxwidth,height=\maxheight,keepaspectratio}
% Set default figure placement to htbp
\makeatletter
\def\fps@figure{htbp}
\makeatother
\setlength{\emergencystretch}{3em} % prevent overfull lines
\providecommand{\tightlist}{%
  \setlength{\itemsep}{0pt}\setlength{\parskip}{0pt}}
\setcounter{secnumdepth}{5}
\usepackage{booktabs}
\ifluatex
  \usepackage{selnolig}  % disable illegal ligatures
\fi
\usepackage[]{natbib}
\bibliographystyle{apalike}

\title{Препораки за репродуцибилно анализирање на податоци}
\author{Теофил Наков, Новица Наков}
\date{11/25/2020}

\begin{document}
\maketitle

{
\setcounter{tocdepth}{1}
\tableofcontents
}
\hypertarget{ux43fux43bux430ux43d}{%
\chapter{План:}\label{ux43fux43bux430ux43d}}

title: Препораки за репродуцибилно ракување, анализирање, и објавување на податоци\\
more realistic title: Препораки за репродуцибилно анализирање на податоци\\
c1: Вовед\\
c2: Алатки\\
c3: Типична анализа (која никој, дури ни авторот, не може да ја повтори)\\
c4: Брза конверзија во репродуцибилна анализа: Скрипта\\
c5: Репродуцибилни документи / извештаи (Rmd)\\
c6: Контрола на изворниот код (git / GitHub)\\
c7: Што ако вашиот колега има Mac, или друга верзија на \texttt{R}? Docker.\\
c8: Заклучок\\
c9: Литература

\hypertarget{intro}{%
\chapter{Вовед}\label{intro}}

Овој текст е обид да се направи преглед и понудат препораки за безбедна пракса при ракување, анализирање, и објавување на податоци. Под „безбедна пракса`` подразбираме чекори кои доколку ги следиме ќе имаме некаква сигурност дека резултатот кој ние го презентираме може било кој да го добие доколку ги користи истите податоци и следи истите процедури кои ги објавуваме заедно со резултатот.

Се чини дека оваа поента е доволно очигледна да некој би рекол дека и воопшто нема потреба да се зборува за ова. Сите се согласуваме со тоа дека кога нашата колешка ќе направи некаква анализа, ние, доколку имаме доволно информации за нејзината анализа, можеме да ја повториме и да ги добиреме \emph{истите} резултати. Но во пракса оваа удобна идеја ретко кога се остварува. Наместо репродуцибилност, кога се обидуваме да повториме некоја анализа, вообичаено е да се соочиме со забуна и фрустрација. Ова често завршува со откажување или правење на анализата од почеток, односно, безполезно и контрапродуктивно губење на време, ресурси, пари, итн.

Горната дефиниција на „безбедна пракса`` (без преправање дека ова е некаква прецизна дефиниција) повикува барем две дополнителни поенти за тоа што подразбира да анализираме податоци и објавувиме резултати на начин што овозможува точно повторување. Имено, објавувањето на резултат мора да биде поддржано со објавување на 1) точните податоци кои биле употребени да се направи анализата, и 2) деталната процедура која била извршена за да се добие резултатот. Подоцна ќе видиме зошто се неопходни овие компоненти (иако веруваме дека е горе-доле очигледно) и на кој начин е најдобро да се презентираат/објават.

Тука е можеби полезно да направиме разлика помеѓу репродуцибилноста која ја опишуваме до сега и еден друг тип на повторливост што има поголема фундаменталност и тежина. Во природните науки, кога зборуваме за нешто што е научна вистина или научно знаење, скоро секогаш тоа се однесува на резултати кои се повторливи и можат да се докажат од различни агли. Дали почнувајќи со различни податоци, или со поставување на нови експерименти, или со употреба поинакви анализи, критично е да дојдеме до истиот резултат за тој да биде прифатен како една компонента од тоа како светот функционира (научна вистина). На пример, дека Земјата е сферична може да се докаже со податоци од циркумнавигација, со мерење на аголот на сончевите зрази на различни географски локации, со движењето на Фуковото нишало, со самото постоење на ГПС навигација, со фотографии од возила во орбита, и така натаму. Значи истиот резултат можеме ги изведеме од многу различни податоци и методи. За разлика од ваквата репликација, фокусот на овој текст е поскромен, и се однесува на далеку полесниот концепт на повторување истата рецепта, со истите составки, за да ја направиме истата пита.

\hypertarget{analogy}{%
\section{Аналогија}\label{analogy}}

Проширување на горната аналогија е добар начин да се запознаеме со главните компоненти на репродуцибилната обработка на податоци. Кога правиме пита имаме состојки (брашно, квасец, зелје, сирење, \ldots), рецепта со неколку чекори (нарасни го квасецот, замеси тесто, насукај кори), и неопходни алатки (лонче, тарун, тепсија, фурна). Кога сакаме да ја направиме истата пита што ја прави баба му на нашиот пријател, треба да ги имаме сите состојки, информации за сите чекори за подготовка, и сите алатки. Ако имаме брашно, вода, и сол ама сме заборавиле квасец, нема да може да замесиме тесто. Ако немаме сукало, ќе треба да тегнеме кори, што секако ќе значи поинаква пита. Значи за да ја повториме питата, потребно е да ги имаме сите неопходни елементи и инфмормации кои ќе овозможат точно повторување на секој чекор. Затоа не е чудно што рецепти за готвење често доаѓаат со слики или видео. Тоа се медиуми побогати со инфромации и атоа поадекватни за пренесување комплексни процеси како правење пита. Безразлика колку детално некој опишал како се сука кора со радиус од 25 cm, со видео од процесот ние добиваме далеку подобро разбирање за процедурата.

Во сферата на анализа за податоци, аналозите на горните компоненти се:

\begin{itemize}
\tightlist
\item
  податоците \textasciitilde{} состојки\\
\item
  изворниот код \textasciitilde{} рецепта и\\
\item
  софтверски пакети од кои зависи нашиот код \textasciitilde{} алатки
\end{itemize}

На пример, ако правиме анализа на невработеност низ Северна Македонија, и имаме една табела со невработеност по општини но сме ги групирале податоците по градови користејќи друга табела, тогаш за точна репликација на нашата анализа треба да ги споделиме двете табели (и брашното и квасецот). Слично, за некој ги направи истите графици за невработеност по град, треба да го споделиме и нашиот код за правење на анализата (рецептата), како и информации за софтверот во кој сме го извршиле тој код (тарунот). Некогаш тарунот не е неопходен за да се направи истата пита, можеби само температурата на фурната е битна, и во ваквите случаи можеме да ја споделиме рецептата без рестрикција за алатката што треба да се користи. Но во други случаи, дури и да се достапни точните податоци и документираниот изворен код, анализата не може да се повтори без некоја специјална алатка. Во овие случаи ние би требало да ги спакуваме и споделиме дури и нашите алатки.

За среќа, аналогијата помеѓу правење пита и повторување на некоја анализа завршува со споделувањето на податоците, кодот, и информации за алаткити. Имајќи пристап до овие компоненти, дури и никогаш да не сте правеле некаква специфична анализа во некој специфичен софтвер, вие сепак ќе можете да ги извршите истите документирани чекори и дојдете до истиот резултат. Повторувањето на анализата не бара пракса, вичност, искуство, и не зависи од тоа колку е влажно брашното. Ако креирањето репродуцибилни документи или анализи ви звучеше како комплициран концепт, се надеваме дека со оваа аналогија станува јасно дека е далеку полесено од обидот да се направи пита како како баба му на пријателот или пица како од ресторан.

\hypertarget{organ}{%
\section{Организација}\label{organ}}

Темата на овој текст е доволно опширна да тука нема ни да се обидеме да дадеме севкупен третмант на сите значајни аспекти. Туку, целта е да дадеме преглед на главните принципи за повторливи анализи и практични примери за главните препораки.

По воведот, прво ќе се запознаеме со софтверските алатки кои ќе ги користиме понатаму (Поглавје 2: Алатки), и ќе посочиме некои од најчестите случаи кога една анализа, дури и да имаме најдобри намери, може да биде ``неповторлива`` (Поглавје 3: Чести проблеми и \ldots{} ???). Понатаму ќе ги разгледаме главните начини на кои постоечки, нерепродуцибилен код за некоја анализа може да се конвертира во репродуцибилна анализа (Поглавје 4: Брза конверзија во репродуцибилна анализа: Скрипта) и за пишување на документи кои ги покажуваат нашето размислување, претпоставки, хипотези, заедно со кодот што го користиме за нивно тестирање и резултатите од тие тестови (Поглавје 5: Репродуцибилни документи / извештаи (Rmd)). За крај, ќе зборуваме за контола на изворниот код со \texttt{git} што ни дава слобода и безбедност за променување и подобрување (Поглавје 6: Контрола на изворниот код (git / GitHub)) и за случаевите кога дури и се околу нашата анализа да е репродуцибилно, некој едноставно не може да ја повтори нашата анализа бидејќи нема пристап до алатките кои ние го користиме (Поглавке 7: Што ако вашиот колега има Mac, или друга верзија на \texttt{R}? Docker.). Текстот го затвораме со заклучок и литература со препорака да во најмала рака ги погледнете ресурсите дадени во библиографијата бидејќи сорджат далеку подетални и ефективни третмани на темата на репродуцибилно ракување, анализирање и објавување на податоци.

\hypertarget{tools}{%
\chapter{Алатки}\label{tools}}

\hypertarget{ux43fux440ux43eux433ux440ux430ux43cux441ux43aux438-ux458ux430ux437ux438ux43a-r}{%
\section{\texorpdfstring{Програмски јазик: \texttt{R}}{Програмски јазик: R}}\label{ux43fux440ux43eux433ux440ux430ux43cux441ux43aux438-ux458ux430ux437ux438ux43a-r}}

Генерално гледано, во денешно време, доколку анализирате податоци тоа вероратно го правите со помош на \texttt{Python} (и пакети како: \texttt{pandas,\ numpy,\ matplotlib}) или \texttt{Р} (со пакети како: \texttt{dplyr,\ data.table,\ ggplot2}). Доколку работите на пониско ниво или вашата работа е поблиска до математика, можеби користите \texttt{c\#} за забрзување на вашиот код, но генерално, ретко кога јазик од типот на \texttt{c\#} сe користи за интерактивна анализа на податоци, графирање, или машинско учење. Конечно понови јазици, како \texttt{Julia} и \texttt{Scala} имаат дополнителни предности со тоа што овозможуваат полесно пишување код во експлоративни сесии како \texttt{Python} ili \texttt{R} но со брзина на компутација што е поблиска до \texttt{c} или \texttt{c\#}. Во секој случај, препораките дадени во овој текст се апликабилни без разлика на програмскиот јазик што го користите.

Во овој текст генерално ќе работиме со софтверски алатки од \texttt{R} екосистемот. \texttt{R} е програмски јазик пред се наменет кон статистички анализи, и иако не е најраспространет или најшироко користен, е од особено значење за анализа на податоци, и следствено репродуцибилна анализа на податоци. Главните причини за употребата на \texttt{R} се дека авторите на овој текст секојдневно користат \texttt{R} и \texttt{Rstudio} и поради тоа што има еден куп дополнителни екстензии во оваа средина кои овозможуваат лесно пишување технички извештаи, научни трудови (\texttt{Rmarkdown}), книги (\texttt{bookdown}), веб апликации (\texttt{shiny}), блогови (\texttt{blogdown}), итн. Самиот овој текст го пишувамe со помош на \texttt{bookdown}, што овозможува неверојатно лесно составање и објавивање на подолги текстови (книги) со поглавја, компјутерски код, и математичка нотација со едно копче во \texttt{Rstudio}.

Во овој текст немаме намера да навлегуваме длабоко во самото кодирање во \texttt{R}. Нашиот приод ќе биде да објасниме одреден принцип со обични зборови и да покажеме како тоа би можело да изгледа со \texttt{R} код. Што значи дури и да не знаете ништо за \texttt{R} би можеле да го следите текстот и употребите препораките во вашиот омилен програмски јазик за обработка на податоци.

Во некој делови ќе користиме материјали кои се специфични за \texttt{R} и \texttt{Rstudio} (на пример \texttt{Rmarkdown}) така да можеби ќе има технички детали кои нема да може да директно да ги примените во \texttt{Jypiter} тетратка или доколку работите во друг текст едитор (\texttt{VScode}). Но повторно, имајќи во предвид дека препораките ќе тежнеат кон тоа како да пишуваме код кој ќе бара минимална интервенција при (ре)анализа на податоци, ваквите аспекти специфични за \texttt{Rstudio} ќе бидат сведени на минимум. Генерално, целта ни е да креираме пакет (фолдер) за правење пита што вклучува се што е неопходно за да нашиот пријател може без многу мислење да ја направи истата пита. За да го олесниме овој процес, треба да се стремиме кон тоа да не користиме специјален тарун (софтвер) за да ги сукаме корите, бидејќи таков тарун можеби нема да биде лесно достапен за нашиот пријател. Со други зборови, повторувањето на нашите резултати генерално не треба да зависи од едиторот за текст кој ние го користиме.

\hypertarget{ux43aux43eux43dux442ux440ux43eux43bux430-ux43dux430-ux438ux437ux432ux43eux440ux435ux43d-ux43aux43eux434-git}{%
\section{\texorpdfstring{Контрола на изворен код: \texttt{git}}{Контрола на изворен код: git}}\label{ux43aux43eux43dux442ux440ux43eux43bux430-ux43dux430-ux438ux437ux432ux43eux440ux435ux43d-ux43aux43eux434-git}}

Контрола на изворниот код, најчесто со \texttt{git}, е платформата што ни дава безбедност и слобода при анализирање на податоци и ревизирање на код и анализи. Во основа, \texttt{git} e систем за \texttt{undo}/\texttt{redo} на стероиди што може да следи две или повеќе верзии на изворниот код (гранки), од два или повеќе компјутери, овозможува соработници од различни локации истовремено да праваt поправки и унапредувања без да си ги пребришуваат промените, и дозволува да се вратите на верзијата од пред два месеци без да ги изгубите меѓувремени промени. Во денешно време е невозможно да се замисли организација која зависи од програмски код за дел од своите фунцкии без употреба на \texttt{git} (или друг систем на контрола на изворен код). Стриктно гледано, \texttt{git} не е неопходен за правење репродуцибилни анализи. Можно е да напишеме и споделиме скрипта која обработува некои податоците без да ги следиме промените на тој код. Но имајќи во предвид дека при обработка на податоци постојано се соочуваме со одлуки и ревизии, веројатно е ќе се најдеме на брег на река каде што ќе сакаме целиот товар да го спакуваме во водоотпорена вреќа пред да ја преминеме реката. \texttt{git} го овозможува токму тоа и е безценета компонента на транспарентната анализа на податоци.

\hypertarget{ux432ux438ux440ux442ux443ux435ux43bux43dux438-ux43aux43eux43dux442ux435ux458ux43dux435ux440ux438-docker}{%
\section{\texorpdfstring{Виртуелни контејнери: \texttt{docker}}{Виртуелни контејнери: docker}}\label{ux432ux438ux440ux442ux443ux435ux43bux43dux438-ux43aux43eux43dux442ux435ux458ux43dux435ux440ux438-docker}}

\hypertarget{ux441ux438ux436ux435}{%
\section{Сиже}\label{ux441ux438ux436ux435}}

Корисно е да ги табулираме овие алатки виз-а-виз компонентите кои ги воведовме претходно во поглавје \ref{analogy}):

\begin{table}

\caption{\label{tab:unnamed-chunk-1}Компонентите на повторување на анализа со соодветните софтверски алатки.}
\centering
\begin{tabular}[t]{l|l|l}
\hline
Аналог & Компонента & Алатка\\
\hline
Состојки & Податоци & csv, MySQL\\
\hline
Рецепта & Изворен код & R, git\\
\hline
Алатки & Зависности на кодот & dplyr, ggplot2\\
\hline
 & Виртуелен контејнер & docker\\
\hline
\end{tabular}
\end{table}

\hypertarget{methods}{%
\chapter{Methods}\label{methods}}

We describe our methods in this chapter.

\hypertarget{applications}{%
\chapter{Applications}\label{applications}}

Some \emph{significant} applications are demonstrated in this chapter.

\hypertarget{example-one}{%
\section{Example one}\label{example-one}}

\hypertarget{example-two}{%
\section{Example two}\label{example-two}}

\hypertarget{final-words}{%
\chapter{Final Words}\label{final-words}}

We have finished a nice book.

  \bibliography{book.bib,packages.bib}

\end{document}
